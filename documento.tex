\documentclass[aspectratio=169]{beamer}

\usepackage[brazil]{babel}
\usepackage[utf8]{inputenc}
\usepackage[T1]{fontenc}

\usetheme{Madrid}

\setbeamertemplate{navigation symbols}{}

\title[Gestão Organizacional]{Gestão Organizacional}

\author[Diego S. C. Nascimento]{Diego Silveira Costa Nascimento}

\institute[IFRN]{
	Instituto Federal de Educação, Ciência e Tecnologia do Rio Grande do Norte\\
	Campus Natal -- Cidade Alta\\
	diego.nascimento@ifrn.edu.br
}

\date[\today]{\today}

\begin{document}

\begin{frame}[plain]
	\includegraphics[scale=0.2]{img/IFRN}
	\titlepage
\end{frame}

\logo{\includegraphics[scale=0.1]{img/IFRN}}

\AtBeginSection[]{
	\begin{frame}
		\frametitle{Sumário}
		\tableofcontents[currentsection]
	\end{frame}
}

\section{Administra\c cão}

\begin{frame}
	\frametitle{Administrar}

	\begin{block}{Origem}
	 Do latim \textit{ad} (dire\c cão) e \textit{minister} (subordina\c cão ou obediência).
	\end{block}
\end{frame}

\begin{frame}
	\frametitle{História}

	\begin{itemize}
		\item 3.000 a.C, Mesopotâmia -- Civiliza\c cão Suméria. Escritura\c cão de opera\c cões comerciais. Primeiros dirigentes e funcionários administrativos profissionais; 
		\item Século XXVI a.C., Egito -- Constru\c cão da grande pirâmide. Planejamento, organiza\c cão e controle elaborados. Primeiro ``livro'' de administra\c cão da história: Deveres do Vizir, gravados em hieróglifos;
		\item VIII a.C. até IV A.D., Império Romano -- Grande organiza\c cão multinacional com institui\c cões administrativas sofisticadas. Diversos tipos de dirigentes, participa\c cão popular no governo, legisla\c cão, estrutura, exército organizado e profissionalizado. Romanos foram precursores das organiza\c cões modernas;
		\item V a.C., Grécia -- Democracia, ética, qualidade, método científico, teoriza\c cão, estratégia, conhecimento e outras ideias fundamentais;
	\end{itemize}
\end{frame}

\begin{frame}
	\frametitle{História}

	\begin{itemize}
		\item 300 a.C, Índia -- Arthashastra de Kautilya, manual de deveres do rei e de seus ministros. Primeiro manual completo de administra\c cão da história;
		\item IV a.C., China -- Sun Tzu, A Arte da Guerra. Manual de estratégia e princípios de comportamento gerencial;
		\item XV - XVI A.D., Itália -- Renascimento. Arsenal de Veneza. Inven\c cão da contabilidade. O Príncipe de Maquiavel. Homem polimático. Capitalismo mercantil. Economia criativa.
		\item XVIII, Inglaterra -- Revolu\c cão industrial;
		\item XIX - XX, Alemanha -- Psicologia experimental. Burocracia;
		\item 1881, Estados Unidos -- Primeira escola de administra\c cão;
		\item Transi\c cão para o Século XX, Estados Unidos -- Início do movimento da administra\c cão científica. Linha do montagem móvel. Movimento da qualidade;
	\end{itemize}
\end{frame}

\begin{frame}
	\frametitle{História}

	\begin{itemize}
		\item 1916, Fran\c ca -- Administra\c cão Industrial e Geral;
		\item Sistema Toyota de Produ\c cão. Administra\c cão enxuta. Modelo Japonês de administra\c cão.
		\item 1969, Estados Unidos -- Funda\c cão do PMI; e
		\item Década de 1980, Estados Unidos -- Reengenharia. Seis Sigmas. Redesenho de processo.
	\end{itemize}
\end{frame}

\begin{frame}
	\frametitle{O que é Teoria Geral da Administra\c cão?}

	\begin{block}{Defini\c cão}
	É o conjunto de conhecimentos a respeito das organiza\c cões e do processo de administrá-las, sendo composta por princípios, proposi\c cões e técnicas em permanente elabora\c cões.
	\end{block}
\end{frame}

\begin{frame}
	\frametitle{Frederick Taylor}

	\begin{itemize}
		\item Liderou o movimento da administra\c cão científica;
		\item Nascido nos Estados Unidos;
		\item Para determinar a melhor forma de executar um trabalho, dividiu cada atividades em movimentos menores e cronometrou cada um deles;
		\item Analisou a a\c cão para eliminar movimentos desnecessários, o que dava a origem ao método mais ágil e eficiente de executar uma tarefa atribuída;
		\item Traz os conceitos de eficácia e eficiência no meio da produ\c cão industrial.
	\end{itemize}
\end{frame}

\begin{frame}
	\frametitle{Eficácia vs Eficiência}

\begin{columns}
	\column{0.5\textwidth}

	\structure{Eficácia}
	\begin{itemize}
		\item Fazer as coisas certas;
		\item Preocupa\c cão com os fins;
		\item Ênfase nos resultados; e
		\item Maximizar os objetivos.
	\end{itemize}
		
	\column{0.5\textwidth}

	\structure{Eficiência}
	\begin{itemize}
		\item Fazer bem as coisas;
		\item Preocupa\c cão com os meios;
		\item Ênfase no processo; e
		\item Ausência de desperdícios.
	\end{itemize}
\end{columns}

\end{frame}

\begin{frame}
	\frametitle{Teoria Científica de Taylor}

	\begin{itemize}
		\item Mecanicismo;
		\item Superespecializa\c cão do trabalhador;
		\item Visão microscópica do homem;
		\item Abordagem de sistema fechado; e
		\item A explora\c cão dos empregados.
	\end{itemize}
\end{frame}

\begin{frame}
	\frametitle{Henry Ford}

	\begin{itemize}
		\item Foi um empresário norte-americano;
		\item Fundador da Ford Motor Company; e
		\item Desenvolveu e implantou a linha de montagem em série.
	\end{itemize}
\end{frame}

\begin{frame}
	\frametitle{Princípios de Ford }

	\begin{itemize}
		\item Princípio da intensifica\c cão;
		\item Princípio da economicidade; e
		\item Princípio da produtividade.
	\end{itemize}
\end{frame}

\begin{frame}
	\frametitle{Henri Fayol}

	\begin{itemize}
		\item Nascido na Fran\c ca;
		\item Foi fundador da teoria clássica da administra\c cão;
		\item Foi o primeiro a tratar a administra\c cão como disciplina para formar lidenra\c cas qualificadas;
		\item Toma como base a busca por máxima eficiência através da visão homem econômico;
		\item Ou seja, considera o homem como racional e com focos racionais;
		\item Criou um sistema para otimizar a gerência dando a cada gerente o seus deveres.
	\end{itemize}
\end{frame}

\begin{frame}
	\frametitle{Fun\c cões Administrativas de Fayol}

	\begin{itemize}
		\item Planejar;
		\item Organiza\c cão;
		\item Dire\c cão; e
		\item Controle.
	\end{itemize}
\end{frame}

\begin{frame}
	\frametitle{Max Weber}

	\begin{itemize}
		\item Nascido na Alemanha;
		\item Fundou o método de análise sociológica; e
		\item Lan\c cou as bases para o estudo das organiza\c cões e da burocracia.
	\end{itemize}
\end{frame}

\begin{frame}
	\frametitle{Princípios de Max Weber}

	\begin{itemize}
		\item Indivíduo;
		\item Ética; e
		\item Rela\c cão social.
	\end{itemize}
\end{frame}

\section{Estratégia}

\begin{frame}
	\frametitle{Estratégia}

	\begin{block}{Defini\c cão}
		``Strategy explains how an organization, faced with competition, will achieve superior performance'' (Michael Porter)
	\end{block}
\end{frame}

\begin{frame}
	\frametitle{Ferramenta de Estratégia de Negócio}

	\begin{itemize}
		\item Missão;
		\item Visão; e
		\item Valores.
	\end{itemize}
\end{frame}

\begin{frame}
	\frametitle{Missão}

	\begin{block}{Defini\c cão}
		É o propósito da empresa existir.
	\end{block}
\end{frame}

\begin{frame}
	\frametitle{Visão}

	\begin{block}{Defini\c cão}
		É a situação em que a empresa deseja chegar (em período definido de tempo).
	\end{block}
\end{frame}

\begin{frame}
	\frametitle{Valores}

	\begin{block}{Defini\c cão}
		São os ideais de atitude, comportamento e resultados que devem estar presentes nos colaboradores e nas relações da empresa com seus clientes, fornecedores e parceiros.
	\end{block}
\end{frame}

\begin{frame}
	\frametitle{Análise de SWOT}

	\begin{block}{Defini\c cão}
		É uma técnica de planejamento estratégico utilizada para auxiliar pessoas ou organizações a identificar forças, fraquezas, oportunidades, e ameaças relacionadas à competição em negócios ou planejamento de projetos.
	\end{block}\vfill
	
	\begin{center}
			\includegraphics[scale=0.2]{img/swot}
	\end{center}
\end{frame}

\end{document}